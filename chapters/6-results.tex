
%---------------------------------------------
%	6. Application & Perform of GNN Algorithm
%---------------------------------------------

%\chapter{Application and Performance of GNN Algorithm}\label{chapter-6}

%The main application of the GNN algorithm on the trackML data, the data preparation, what data was exactly used and why, the ML classification algorithm used to build the edge connections. The main results we get from application of this algorithm, the track reconstruction efficiency, the purity metrics, computational performance. Comparison with other algorithms.

%\section{Data Preparation}

%Talk specifically about the TrackML data here and how it was prepared.

%nips-2018-competation
%Data.
%We used the fast (10s per event) and accurate simulation engine ACTS4 [6] to generate the challenge data. It allowed us to generate realistic data emulating a full Silicon LHC detector (see Fig 3), while providing us with the ground truth of particle trajectory membership. Thus, for each event we obtained the “detected” 3D points coordinates (and additional features), and, as ground truth, the list of points associated to each track. There is a one to one relationship between the true 3D points and the reconstructed ones.

%\section{Results}
%\subsection{Performance Evaluation}
%\subsection{Analysis and Optimisation}
%\subsection{Comparison and Outlook}
%\subsection{Execution time}


%---------------------------------------------
%	6. Application & Perform of GNN Algorithm
%---------------------------------------------

\chapter{Application and Performance of GNN Algorithm}
\label{chapter-gnn-results}

The main application of the GNN algorithm on the trackML data, the data preparation, what data was exactly used and why, the ML classification algorithm used to build the edge connections. The main results we get from application of this algorithm, the track reconstruction efficiency, the purity metrics, computational performance. Comparison with other algorithms.

\section{Data Preparation}

Talk specifically about the TrackML data here and how it was prepared.

%nips-2018-competation
%Data.
%We used the fast (10s per event) and accurate simulation engine ACTS4 [6] to generate the challenge data. It allowed us to generate realistic data emulating a full Silicon LHC detector (see Fig 3), while providing us with the ground truth of particle trajectory membership. Thus, for each event we obtained the “detected” 3D points coordinates (and additional features), and, as ground truth, the list of points associated to each track. There is a one to one relationship between the true 3D points and the reconstructed ones.

\section{Results}
Track reconstruction efficiency, track purity, particle purity
\section{Performance Evaluation}
\section{Analysis and Optimisation}
\section{Comparison and Outlook}
\section{Execution time}
\section{Conclusions}