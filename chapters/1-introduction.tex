%---------------------
%	1. Introduction
%---------------------

%\doublespacing
%\setcounter{section}{0}
\chapter{Introduction}

%\begin{itemize}
%\item Introduction to the thesis
%\item The overall narrative and where this thesis is going
%\item "The bigger picture"
%\item Particle detector experiments in general, the increase in luminosity, the physics motivation, the challenges to tracking that arise from this - the CPU usage, optimisations etc, the pileup, emphasize the i importance of optimisations in tracking
%\item The importance of ML is tackling these challenges
%\item Graph networks as a solution which is used in tracking, traditionally multi-layered perceptrons are trained, but this approach is unique
%\item First it is important to find good edge connections in the network (QT task) and then find good track candidates within the network that can be extracted (GNN)
%\end{itemize}

%Background and impact.
%Importance of the problem and anticipated impact.

%nips-2018-competition
%Our challenge program inserts itself in a bigger effort of the Atlas collaboration (one of the three experiments analyzing data collected at CERN on the Large Hadron Collider– LHC) to use Machine Learning to assist high energy physicists in discovering and characterizing new particles.
%In the LHC, proton bunches (beams) circulates and collide at high energy. Each beam- beam collision (further called an event) produces a firework of new particles (figure 2). To identify the types and measure the kinematic properties of these particles, a complex apparatus, the detector records the small energy deposited by the particles when they impact well-defined locations in the detector.
%The tracking problem refers to reconstructing the trajectories of the particles from the information recorded by the detector. The augmentation of data throughput creates a major scaling bottleneck for the associated pattern recognition-tracking task. Thus, current methods [5] will soon become obsolete and there is an urgent need for novel algorithms.



%\end{document}

