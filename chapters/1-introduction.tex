%---------------------
%	1. Introduction
%---------------------

%\doublespacing
%\setcounter{section}{0}
\chapter{Introduction}

\setlength\parindent{0pt}

This thesis describes efforts to improve the understanding of track reconstruction within high energy charged particle detector experiments via graph based methods. This is primarily achieved through the development of algorithms used to construct graph networks with compatible edge connections, as well as the extraction of good track candidates from graph networks. 

This thesis is structured in the following manner:

Chapter \ref{chapter-2} describes the ATLAS detector and the CERN accelerator complex. Details of future upgrades to the particle detector experiment are also given.

Chapter \ref{chapter-3} provides an overview of charged particle trajectory (track) reconstruction in silicon based particle detectors. This chapter also presents the TrackML detector model, a realistic detector model simulation and online Machine Learning challenge, as well as an introduction to graph network structures and track reconstruction using graph based architectures.

Chapter \ref{chapter-4} describes the development and application of a machine learning based algorithm which predicts if a pair of hits belong to the same track given input hit features. This chapter showcases a methodology that can be used for graph building within track reconstruction.

Chapter \ref{chapter-5} encapsulates the development of a novel pattern recognition algorithm utilising graph neural network (GNN) architectures and Kalman filters (KF) in order to identify and extract track candidates in a silicon based Pixel detector. The GNN track reconstruction algorithm is outlined in this chapter.

Chapter \ref{chapter-6} provides an overview of the application and performance of the GNN algorithm on the TrackML detector model, as well as the challenges faced during development. Preliminary investigations into improvements and an outlook to software enhancement are also discussed.

Chapter \ref{chapter-7} contains some concluding remarks.

The author’s contribution to the work presented in this thesis is as follows.

The author was an active member of the Inner Detector (ID) trigger group at the ATLAS experiment throughout their PhD, starting with their qualification task on developing a machine learning based classifier for measurement-to-track association to predict if a pair of hits belong to the same track for the ATLAS ID trigger. The author has presented at the Advanced Computing and Analysis Techniques (ACAT) online conference in Daejeon Korea, in 2021 and at the Institute of Physics (IoP) High Energy Particle Physics and Astroparticle Physics (HEPP and APP) conference in 2022. The author is published in the Journal of Physics: Conference Series \cite{Lad_2023} and the software is implemented in the optimisation of the HLT ID track seeding software for ATLAS Run-3 and beyond \cite{Grandi:2728111, Long:2813981}. The author also played a key role in contributing to the ID trigger validation tasks. Such tasks involved weekly reprocessing for validation of upcoming software releases, in order to catch signs of rare bugs, test online monitoring and validate the output of the High Level Trigger (HLT) algorithms.

The author presented the development and applications of the GNN-based algorithm for track reconstruction at the 2022 Connecting the Dots (CTD) conference at the University of Princeton USA, and is currently under review for publication in the Springer Journal: Computing for Software and Big Science \cite{Lad_2023_gnn}. The author has also presented at several workshops, including the dedicated GNN Google DeepMind seminar, held at UCL in 2023.

