%------------------------------------
%	7. Summary & Outlook
%------------------------------------


\chapter{Summary and Outlook}
\label{chapter-8}

% 1. Intro & background

The current approaches to track reconstruction have proven to be powerful, however they are not designed to handle the conditions presented by extreme environments, such as the HL-LHC. Methods based on seeded track following and the combinatorial KF are very effective, however they do not scale well as the seed number grows. As discussed in Chapter \ref{chapter-2}, the task of track reconstruction becomes increasingly difficult as the luminosity increases. The collision rate and hence hit occupancy will also increase significantly during future upgrades of the LHC program, as well as for other silicon detectors. Therefore, novel and precise tracking methodologies, as well as efficient use of computing power, are paramount in the reconstruction of particle trajectories. Simultaneous consideration of reducing the overall environmental impact and maintaining the ability to reconstruct tracks with minimal loss in efficiency, is at the heart of the problem. As such, enhancements in track reconstruction algorithms are imperative for the future growth of particle detector experiments. 

The use of ML algorithms for track reconstruction has provided a route to develop methodologies that can save vast amounts of CPU resources. An original approach has been explored in great detail in this thesis utilising GNN architectures for efficient track reconstruction. We begin with a procedure to construct a graph network from a collision event, followed by an algorithm designed to iteratively identify and extract track candidates. 



% 2. Section on graph building and the ML classification predictor algorithm developed…

When building graph networks for track reconstruction, an essential aspect to consider is the ability to identify compatible hit connections to build graph edges. Chapter \ref{chapter-4} successfully demonstrates a methodology to achieve this, whereby a ML-based algorithm is developed to predict if a pair of hits belong to the same track. This process is essential for efficient construction of graph networks to be used for track reconstruction, as it successfully reduces the number of fake edges and hence increases the accuracy in predicting compatible hit-pairs. This simultaneously reduces the number of computations required during the combinatorial stages of tracking algorithms, as well as propagating these benefits to downstream algorithms within the pipeline of track reconstruction software.

The implementation of the ML-based predictor begins with the exploitation of input hit features in its design, namely Pixel cluster width and inverse track inclination. The algorithm employs the use of Bayesian analysis to discriminate between hit-pair classes, alongside the addition of Kernel Density Estimates to provide sophisticated estimates of the likelihood functions. The implementation of the classifier’s predictions as a LUT is essential here to reduce computational overheads, which bodes well for use in realistic detectors.

The application of the ML-based classifier was used for seed selection in the ATLAS ID. It has successfully optimised the HLT ID track seeding software for ATLAS Run-3 and beyond, by reducing the number of fake track seeds and provided significant savings in computing resources. The trained predictor in the form of a LUT yields 2.3$\times$ speed-up with minimal loss in efficiency (1.1\%) at $< \mu >$= 80 compared with the standard trigger tracking. 

%The developed ML framework is advantageous for many reasons.




% end
As future upgrades to particle detectors will be problematic for silicon tracking detectors, where hit occupancy is the highest, it is essential that computing resource use is also reduced, whilst maintaining the capability to reconstruct tracks with minimal efficiency loss. This consideration reduces the environmental impact by providing vast savings in computational energy and simultaneously yields a sustainable advancement in algorithmic development.
