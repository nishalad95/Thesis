%--------------------------------------------------
%	Chapter 5. GNN Pattern Recognition Algorithm
%--------------------------------------------------

\chapter{Graph Neural Network Pattern Recognition Algorithm}\label{chapter-5}

% The excitation and inhibition rules of individual GNN nodes will be designed to facilitate the “simple-to-complex” approach for “hits-to-tracks” association, such that the network starts with relatively “easy” areas of an event (low hit density) and gradually progresses towards more complex areas (high hit density). 

This work aims at implementing a realistic GNN-based algorithm that can be deployed in an real particle detector experiments.

The main algorithm - the overview, its purpose, the main aims, the different stages, the different techniques used, hyper parameter tuning, implementation of KFs - linear with OU model for extraction and parabolic model for information aggregation/extrapolation.

Mention that this work is presented in XXX conference (and reference).

\section{Algorithm Overview}

\section{Network Initialization}
\subsection{Track State Estimates}
\subsection{Molière Theory of Multiple Scattering}

% moliere theory links:

%https://gray.mgh.harvard.edu/attachments/article/337/Techniques%20of%20Proton%20Radiotherapy%20(06)%20Multiple%20Scattering.pdf

%https://pdg.lbl.gov/2005/reviews/passagerpp.pdf

\section{Gaussian Mixture Reduction}
\subsection{Clustering and KL divergence}

\section{Information Aggregation}
\subsection{Message Passing}
\subsection{Extrapolation and Validation}
\subsection{Linear and Parabolic model}

\section{Updating Network State}

\section{Track Extraction}
\subsection{Implementation of KFs}
\subsection{CCA}
\subsection{Merging Close Proximity Nodes}
\subsection{Community Detection}
%Community Detection: divides nodes into various clusters based on edge structure. It learns from edge weights, and distance and graph objects similarly. 



\section{Conclusions}
