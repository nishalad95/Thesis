%--------------------------------------------------
%	Chapter 5. GNN Pattern Recognition Algorithm
%--------------------------------------------------

\chapter{Graph Neural Network Pattern Recognition Algorithm}\label{chapter-5}

%encapsulates the development of a novel pattern recognition algorithm utilising graph neural network (GNN) architectures and Kalman filters in order to identify and extract track candidates in a silicon based Pixel detector. The GNN track reconstruction algorithm is outlined in this chapter.


% The excitation and inhibition rules of individual GNN nodes will be designed to facilitate the “simple-to-complex” approach for “hits-to-tracks” association, such that the network starts with relatively “easy” areas of an event (low hit density) and gradually progresses towards more complex areas (high hit density). 

This work aims at implementing a realistic GNN-based algorithm that can be deployed in an real particle detector experiments.
The main algorithm - the overview, its purpose, the main aims, the different stages, the different techniques used, hyper parameter tuning, implementation of KFs - linear with OU model for extraction and parabolic model for information aggregation/extrapolation.
Mention that this work is presented in XXX conference (and reference).



\section{Algorithm Overview}
Introduction to the main iterations and the stages, as well as the key concepts and techniques used within the algorithm.



\section{Network Initialization}
\begin{itemize}
\item Using networkX python package?
\item Event conversion and graph construction nodes and edges?
\item Gaussian mixture equations
\item Track State equations, covariances, derivations, r-z plane and transverse plane
\item edges having priots and weights (strength and compatibility) - because of Gaussian mixture
\item edges act as bidrectional conduits - passing information both ways between nodes
\end{itemize}

\subsection{Track State Estimates}
\begin{itemize}
\item In both xy and rz planes, joint track state estimate
\item parabolic model - in local coordinate system of node itself
\item rz joint vector - tau parameter, inverse track inclination
\item how were the covariances dervied and the sigma errors chosen
\end{itemize}

\subsection{Molière Theory of Multiple Scattering}
\begin{itemize}
\item highland formula and handling the error/effects due to multiple scattering for the barrel and endcap in slightly different ways
\end{itemize}
% moliere theory links:
%https://gray.mgh.harvard.edu/attachments/article/337/Techniques%20of%20Proton%20Radiotherapy%20(06)%20Multiple%20Scattering.pdf
%https://pdg.lbl.gov/2005/reviews/passagerpp.pdf



\section{Gaussian Mixture Reduction}
\begin{itemize}
    \item GMR theory
    \item Clustering, KL divergence
    \item Mahalanobis distance and merging of states
    \item training the optimal threshold using an SVM
    \item feeding in using a fast look-up table
\end{itemize}



\section{Information Aggregation}
\begin{itemize}
    \item Message Passing
    \item Extrapolation and Validation
    \item Linear and Parabolic model - 2 different extrapolations for xy componenets of state vector and rz componenet, illustrations here
    \item Kalman Filter Update, OU process for correlated noise
\end{itemize}



\section{Updating Network State}
\begin{itemize}
    \item As edges get turned on and off - local neighbourhoods change, track states and covariances need updating
    \item reweighting - calculation of Gaussian likelihood in order to update the weights
    \item deactivation of any weights that are below a tuned threshold
    \item recalculation of priors
    \item Gaussian mixture update local to each neighbourhood
\end{itemize}



\section{Track Extraction}
\begin{itemize}
    \item CCA
    \item criteria for a good track candidate
    \item Implementation of KFs/ KF track fit
    \item Merging Close Proximity Nodes
    \item Community Detection
\end{itemize}
%Community Detection: divides nodes into various clusters based on edge structure. It learns from edge weights, and distance and graph objects similarly. 



\section{Implementation of KFs}
\begin{itemize}
\item emphasis on the use of KFs both in information aggregation stage and in track extraction, both are implemented in different ways, is a useful and unique part to this algorithm
\end{itemize}



\section{Application and Performance on a Simple Toy MC Model}
\begin{itemize}
\item Application on a simple toy mc model and heat network - track extracted and metrics
\end{itemize}



\section{Conclusions}
