% Intro, Tracking and environmental impact:
This thesis details research in experimental particle physics. The primary contributions are on the improvement of pattern recognition algorithms for reconstruction of charged particle trajectories (tracks) at the ATLAS detector at the Large Hadron Collider. Track reconstruction is a highly computationally intensive task. Therefore, finding an alternative solution that proposes a more efficient method utilizing Graph Neural Networks (GNNs), has a large positive impact for the entire research program at CERN. This research can lead to huge environmental benefits in saving vast amounts of energy and computational resource, during future upgrades of all particle detector experiments.

% GNN paragraph:
This thesis is also an advancement in knowledge of GNN architectures. GNNs have become incredibly widespread in different domains, for instance the development of pharmaceutical drugs by modelling molecular interactions and understanding social-media networks. Exploration into GNNs is beneficial, as knowledge of this technique can be propagated into many areas of science and society.

% Wider reach of physics research at the LHC and CERN:
In general, the research at CERN does find indirect applications in the form of associated technological developments within different fields. The techniques developed include the World Wide Web, high-field magnet technology in MRI and cloud computing. Therefore, fundamental physics as a method of solving difficult and novel problems, can be seen as a way to generate innovative technologies.

% Research and data science in general, interest in scientific research and end:
Working in the field also helps to train skilled researchers, who can be redeployed to other areas of society to tackle various problems. In this thesis, advanced statistical and data science methods are employed. The training of individuals highly skilled in these areas has a sustained positive economic impact. Finally, the work carried out at the CERN is widely publicised. Support of and interest in physics research helps to generate excitement about science and technology.
