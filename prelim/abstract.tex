%Future upgrades to modern high-energy particle detectors will pose considerable challenges for traditional particle track reconstruction methods. Within the past two years, architectures that operate on Graph Neural Networks (GNN) have shown high degrees of promise. Presented here is a novel approach to GNN-based track finding which uses Gaussian mixture techniques to describe track parameter estimates. Unlike traditional methodologies whereby Multi-Layered Perceptrons (MLPs) are employed, the proposed GNN architecture leverages the Kalman filter as a mechanism for information aggregation, in order to iteratively improve the precision of track parameters, as well as for extraction of track candidates compatible with particle motion model. The excitation and inhibition rules of individual edge connections are designed to facilitate the “simple-to-complex” approach for “hits-to-tracks” association, such that the network starts with low hit density regions of an event and gradually progresses towards more complex areas. This thesis focuses on the track finding algorithm development and its application on the publicly available dataset designed for the Kaggle TrackML challenge. The preliminary results related to track reconstruction efficiency and purity metrics are presented and discussed. The ultimate aim of this work is to develop a realistic GNN-based algorithm for fast track finding that can be deployed in future high-luminosity phases of particle detector experiments.


% Talk about the ML predictor classifier to build compatible edge connections and to construct the graph network, once the network is constructed then we move to the GNN for pruning.
% Give preliminary key results 